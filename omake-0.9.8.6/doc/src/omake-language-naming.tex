%%%%%%%%%%%%%%%%%%%%%%%%%%%%%%%%%%%%%%%%%%%%%%%%%%%%%%%%%%%%%%%%%%%%%%%%
% Description
%
\chapter{Variables and Naming}
\label{chapter:naming}
\cutname{omake-language-naming.html}

During evaluation, there are three different kinds of namespaces.  Variables can be \emph{private},
or they may refer to fields in the current \emph{this} object, or they can be part of the
\emph{global} namespace.  The namespace can be specified directly by including an explicit qualifier
before the variable name.  The three namespaces are separate; a variable can be bound in one or more
simultaneously.

\begin{verbatim}
    # private namespace
    private.X = 1
    # current object
    this.X = 2
    # public, globally defined
    global.X = 3
\end{verbatim}

\section{private.}
\label{section:private}\index{private.}

The \verb+private.+ qualifier is used to define variables that are private to the current file/scope.
The values are not accessible outside the scope.  Private variables are statically (lexically) scoped.

\begin{verbatim}
    Obj. =
       private.X = 1

       print() =
          println(The value of X is: $X)

    # Prints:
    #    The private value of X is: 1
    Obj.print()

    # This is an error--X is private in Obj
    y = $(Obj.X)
\end{verbatim}

In addition, private definitions do not affect the global value of a variable.

\begin{verbatim}
   # The public value of x is 1
   x = 1

   # This object uses a private value of x
   Obj. =
       private.x = 2

       print() =
          x = 3
          println(The private value of x is: $x)
          println(The public value of x is: $(public.x))
          f()

   # Prints:
   #    The private value of x is: 3
   #    The public value of x is: 1
   Obj.print()
\end{verbatim}

Private variables have two additional properties.

\begin{enumerate}
\item Private variables are local to the file in which they are defined.
\item Private variables are not exported by the \verb+export+ directive, unless they are
  mentioned explicitly in the \verb+export+ directive.

  \begin{verbatim}
       private. =
          FLAG = true

       section
          FLAG = false
          export

       # FLAG is still true
       section
          FLAG = false
          export FLAG

       # FLAG is now false
  \end{verbatim}
\end{enumerate}

\section{this.}
\index{this.}

The \verb+this.+ qualifier is used to define fields that are local to an object.
Object variables are dynamically scoped.

\begin{verbatim}
    X = 1
    f() =
       println(The public value of X is: $(X))

    # Prints:
    #    The public value of X is: 2
    section
       X = 2
       f()

    # X is a protected field in the object
    Obj. =
       this.X = 3

       print() =
          println(The value of this.X is: $(X))
          f()

    # Prints:
    #    The value of this.X is: 3
    #    The public value of X is: 1
    Obj.print()

    # This is legal, it defines Y as 3
    Y = $(Obj.X)
\end{verbatim}

In general, it is a good idea to define object variables as protected.  The resulting code is more
modular because variables in your object will not produce unexpected clashes with variables defined
in other parts of the project.

\section{global.}
\label{section:global}\index{global.}

The \verb+global.+ qualifier is used to specify global dynamically-scoped variables.  In the following
example, the \verb+global.+ definition specifies that the binding \verb+X = 4+ is to be dynamically
scoped.  Global variables \emph{are not} defined as fields of an object.

\begin{verbatim}
    X = 1
    f() =
       println(The global value of X is: $(X))

    # Prints:
    #    The global value of X is: 2
    section
       X = 2
       f()

    Obj. =
       this.X = 3

       print() =
          println(The protected value of X is: $(X))
          global.X = 4
          f()

    # Prints:
    #    The protected value of X is: 3
    #    The global value of X is: 4
    Obj.print()
\end{verbatim}

\section{protected.}
\label{section:protected}\index{protected.}

In OMake 0.9.8, \verb+protected+ is a synonym for \verb+this+.

\begin{verbatim}
    osh>protected.x = 1
    - : "1" : Sequence
    osh>value $(this.x)
    - : "1" : Sequence
\end{verbatim}
%
In 0.9.9, this will change, so that the qualifier \verb+protected+ means (in 0.9.9) that a variable
is local to the current object or file, and may not be accessed outside it.

\section{public.}
\label{section:public}\index{public.}

In OMake 0.9.8, \verb+public+ is a synonym for \verb+global+.

\begin{verbatim}
    osh>public.x = 1
    - : "1" : Sequence
    osh>value $(global.x)
    - : "1" : Sequence
\end{verbatim}
%
In 0.9.9, this will change, so that the qualifier \verb+public+ means (in 0.9.9) that a variable
is to be accessible from outside the current file or object.

\section{Qualified blocks}

If several qualified variables are defined simultaneously, a block form of qualifier can be defined.
The syntax is similar to an object definition, where the name of the object is the qualifier itself.
For example, the following program defines two private variables \verb+X+ and \verb+Y+.

\begin{verbatim}
    private. =
        X = 1
        Y = 2
\end{verbatim}
%
The qualifier specifies a default namespace for new definitions in the block.  The contents of the
block is otherwise general.

\begin{verbatim}
    private. =
        X = 1
        Y = 2
        public.Z = $(add $X, $Y)
        # Prints "The value of Z is 3"
        echo The value of Z is $Z
\end{verbatim}
%
Stylistically, it is usually better to avoid large qualified blocks because the qualifier status can
be easy to forget.  For example, consider the following fragment.

\begin{verbatim}
    private. =
        # Large code sequence
        ...
        # build foo.o with -g option (ERROR)
        CFLAGS = -g
        foo.o:
\end{verbatim}
%
In this case, the programmer probably forgot that the definition of the variable \verb+CFLAGS+ is in
the \verb+private+ block, so a fresh variable \verb+private.CFLAGS+ is being defined, not the global
one.  The target \verb+foo.o+ does \emph{not} use this definition of \verb+CFLAGS+.

\section{declare}
\label{section:declare}\index{declare}

When a variable name is unqualified, its namespace is determined by the most recent definition or
declaration that is in scope for that variable.  We have already seen this in the examples, where a
variable definition is qualified, but the subsequent uses are not qualified explicitly.  In the
following example, the first occurrence of \verb+$X+ refers to the \emph{private} definition,
because that is the most recent.  The public definition of \verb+X+ is still \verb+0+, but the
variable must be qualified explicitly in order to access the public value.

\begin{verbatim}
    public.X = 0
    private.X = 1

    public.print() =
        println(The value of private.X is: $X)
        println(The value of public.X is: $(public.X))
\end{verbatim}
%
Sometimes it can be useful to declare a variable without defining it.  For example, we might have a
function that uses a variable \verb+X+ that is to be defined later in the program.  The
\verb+declare+ directive can be used for this.

\begin{verbatim}
    declare public.X

    public.print() =
        println(The value of X is $X)

    # Prints "The value of X is 2"
    X = 2
    print()
\end{verbatim}

Finally, what about variables that are used but not explicitly qualified?  In this case, the following rules are used.

\begin{itemize}
\item If the variable is a function parameter, it is private.
\item If the variable is defined in an object, it is qualified with \verb+this.+.
\item Otherwise, the variable is public.
\end{itemize}

% -*-
% Local Variables:
% Mode: LaTeX
% fill-column: 100
% TeX-master: "paper"
% TeX-command-default: "LaTeX/dvips Interactive"
% End:
% -*-
